\begin{longtable}{||l l l l l l||}%
\hline%
\rowcolor{GroundColor}%
&Pokémon&Level Range&Held Item&Rarity Tier&Spawn Times\\%
\hline%
\endhead%
\hline%
\rowcolor{GroundColor}%
\includegraphics[width=0.02\textwidth]{pokemon/Bellsprout.png}&Bellsprout&21{-}22&&\textcolor{black}{%
Common%
}&\textcolor{yellow}{Morn}  \textcolor{orange}{Day}\\%
\hline%
\rowcolor{GroundColor}%
\includegraphics[width=0.02\textwidth]{pokemon/Gastly.png}&Gastly&21{-}22&&\textcolor{black}{%
Common%
}&\textcolor{blue}{Night}\\%
\hline%
\rowcolor{GroundColor}%
\includegraphics[width=0.02\textwidth]{pokemon/Growlithe.jpg}&Growlithe&21{-}22&Rawst Berry&\textcolor{OliveGreen}{%
Uncommon%
}&\textcolor{yellow}{Morn}  \textcolor{orange}{Day}  \textcolor{blue}{Night}\\%
\hline%
\rowcolor{GroundColor}%
\includegraphics[width=0.02\textwidth]{pokemon/Hoothoot.png}&Hoothoot&21{-}22&&\textcolor{black}{%
Common%
}&\textcolor{blue}{Night}\\%
\hline%
\rowcolor{GroundColor}%
\includegraphics[width=0.02\textwidth]{pokemon/Houndour.png}&Houndour&21{-}22&&\textcolor{RedOrange}{%
Rare%
}&\textcolor{yellow}{Morn}  \textcolor{blue}{Night}\\%
\hline%
\rowcolor{GroundColor}%
\includegraphics[width=0.02\textwidth]{pokemon/Ledyba.png}&Ledyba&21{-}22&&\textcolor{black}{%
Common%
}&\textcolor{yellow}{Morn}\\%
\hline%
\rowcolor{GroundColor}%
\includegraphics[width=0.02\textwidth]{pokemon/Nidoran F.png}&Nidoran F&21{-}22&&\textcolor{black}{%
Common%
}&\textcolor{yellow}{Morn}  \textcolor{orange}{Day}\\%
\hline%
\rowcolor{GroundColor}%
\includegraphics[width=0.02\textwidth]{pokemon/Nidoran M.png}&Nidoran M&21{-}22&&\textcolor{black}{%
Common%
}&\textcolor{yellow}{Morn}  \textcolor{orange}{Day}\\%
\hline%
\rowcolor{GroundColor}%
\includegraphics[width=0.02\textwidth]{pokemon/Shinx.png}&Shinx&21{-}22&Electirizer&\textcolor{RedOrange}{%
Rare%
}&\textcolor{orange}{Day}\\%
\hline%
\rowcolor{GroundColor}%
\includegraphics[width=0.02\textwidth]{pokemon/Stantler.png}&Stantler&21{-}22&&\textcolor{black}{%
Common%
}&\textcolor{blue}{Night}\\%
\hline%
\rowcolor{GroundColor}%
\includegraphics[width=0.02\textwidth]{pokemon/Sudowoodo.png}&Sudowoodo&21{-}22&Hard Stone&\textcolor{RedOrange}{%
Rare%
}&\textcolor{yellow}{Morn}  \textcolor{orange}{Day}  \textcolor{blue}{Night}\\%
\hline%
\rowcolor{GroundColor}%
\includegraphics[width=0.02\textwidth]{pokemon/Vulpix.png}&Vulpix&21{-}22&Rawst Berry&\textcolor{OliveGreen}{%
Uncommon%
}&\textcolor{yellow}{Morn}  \textcolor{orange}{Day}  \textcolor{blue}{Night}\\%
\hline%
\caption{Wild Pokémon in Route 36 (Land)}%
\label{tab:Route36Land}%
\end{longtable}